\documentclass[twocolumn]{article}

\usepackage[a4paper, top    = 2.5cm, 
                     bottom = 2.0cm, 
                     left   = 2.5cm, 
                     right  = 2.0cm]{geometry}

\usepackage[brazil]{babel}
\usepackage[T1]{fontenc}
\usepackage{microtype}

\usepackage{indentfirst}

\usepackage{float}

\usepackage{tikz}
\tikzset{>=latex} % for LaTeX arrow head

\title{AVR Pong}
\author{KrpytonPlusPlus}

\begin{document}

\maketitle

\section{Introdução}
    Esse projeto visa projetar um clone do jogo \textsl{Pong} em um microcontrolador \textsl{avr} utilizando o padrão de comunicação \textsl{vga} para exibir a imagem em um monitor.

    Para isso foi dimensionado os ciclos de \textsl{cpu} para comportar o envio das informações de varredura dos \textsl{pixels} e os ciclos sobressalentes foram utilizados para carregar os dados para os \textsl{buffers} com as cores de cada pixel em uma linha (preto ou branco) e fazer o tratamento das iterações do jogo.

    Buscando facilitar os cálculos e aumentar a compatibilidade com os monitores foi utilizado um sistema de \textsl{clock} de \(25{,}175MHz\) para o microcontrolador superior ao limite estipulado pelo fabricante no manual, porém como vai ser visto adiante é a mesma frequência de sinal utilizada pelo monitor para gerar a imagem.

\section{VGA (Video Graphics Array)}
    \subsection{Regiões de varredura}
    
    \begin{figure}[H]
\centering
\resizebox{0.45\textwidth}{!}
{
\begin{tikzpicture}[x = .1mm, y = .1mm]
    
    \def\hvisiblearea {640}
    \def\vvisiblearea {480}
    
    \def\hfrontporch {16 * 2} % alterado para ficar mais estetico
    \def\vfrontporch {10 * 2} %

    \def\hsyncpulse {96}
    \def\vsyncpulse {2 * 4} % alterado para ficar mais estetico

    \def\hbackporch {48}
    \def\vbackporch {33}

    \begin{scope}[xshift = 5mm, local bounding box = pulse horizontal]
        \draw [black] (0, 0) -- ++(\hvisiblearea + \hfrontporch, 0) -- ++(0, -3.5mm) -- ++(\hsyncpulse, 0) -- ++(0, 3.5mm) -- ++(\hbackporch, 0);
        \node [above, font = \small] at (\hvisiblearea / 2, 0) {Pulso de sincronismo horizontal};
    \end{scope}

    \begin{scope}[yshift = -5mm, local bounding box = pulse vertical]
        \draw [black] (0, 0) -- ++(0, -\vvisiblearea - \vfrontporch) -- ++(3.5mm, 0) -- ++(0, -\vsyncpulse) -- ++(-3.5mm, 0) -- ++(0, -\vbackporch);
        \node [above, font = \small, rotate = 90] at (0, -\vvisiblearea / 2) {Pulso de sincronismo vertical};
    \end{scope}
    
    \begin{scope}[xshift = 5mm, yshift = -5mm, local bounding box = screen]
        % visible area
        \draw [black] (0, -\vvisiblearea) -- ++(\hvisiblearea, 0) -- ++(0, \vvisiblearea);
        % front porch
        \draw [black] (0, -\vvisiblearea - \vfrontporch) -- ++(\hvisiblearea + \hfrontporch, 0) -- ++(0, \vvisiblearea + \vfrontporch);
        % sync pulse
        \draw [black] (0, -\vvisiblearea - \vfrontporch - \vsyncpulse) -- ++(\hvisiblearea + \hfrontporch + \hsyncpulse, 0) -- ++(0, \vvisiblearea + \vfrontporch + \vsyncpulse);
        % back porch
        \draw [black] (0, 0) rectangle ++ (\hbackporch + \hvisiblearea + \hfrontporch + \hsyncpulse, -\vbackporch - \vvisiblearea - \vfrontporch - \vsyncpulse);
    
        %
        \node[above right = 0.8mm, font = \small, rotate = 90] at (\hvisiblearea, -\vvisiblearea) {\textit{Visible Area}};
        \node[above right = 0.8mm, font = \small, rotate = 90] at (\hvisiblearea + \hfrontporch, -\vvisiblearea) {\textit{Front Porch}};
        \node[above right = 0.8mm, font = \small, rotate = 90] at (\hvisiblearea + \hfrontporch + \hsyncpulse, -\vvisiblearea) {\textit{Sync Pulse}};
        \node[above right = 0.8mm, font = \small, rotate = 90] at (\hvisiblearea + \hfrontporch + \hsyncpulse + \hbackporch, -\vvisiblearea) {\textit{Back Porch}};
    \end{scope}

    \draw [gray, fill opacity = 0.5] (5mm,   -7.5mm) -- ++(0,    5mm);
    \draw [gray, fill opacity = 0.5] (7.5mm, -5mm)   -- ++(-5mm, 0);
\end{tikzpicture}
}
\caption{Regiões de varredura da tela}
\end{figure}

    No monitor normalmente a \textsl{Back Porch} vem antes da \textsl{Visible Area}, porém para facilitar a apresentação e por ser um sinal cíclico foi apresentada no final.

    Para facilitar a visualização as dimensões de cada região não estão proporcionais aos valores reais.
    
    \begin{table}[H]
        \centering
        \begin{tabular}{r c c}                                            \cline{2-3}
                                  & \multicolumn{2}{c}{\textsl{Pixels}} \\\cline{2-3}
                                  & Horizontal      & Vertical          \\\hline
            \textsl{Back Porch}   & 48              & 33                \\
            \textsl{Visible Area} & 640             & 480               \\
            \textsl{Front Porch}  & 16              & 10                \\
            \textsl{Sync Pulse}   & 96              & 2                 \\
            Total                 & 800             & 525               \\\hline
        \end{tabular}
        \caption{Dimensão horizontal e vertical em \textsl{pixels} da tela para cada região de varredura}
        \label{tab:dimensions_scanning_regions}
    \end{table}

    Normalmente os dados apresentados na tabela \ref{tab:dimensions_scanning_regions} são dados em micro segundos e quantidade de linhas, para a dimensão horizontal e vertical, respectivamente.
    
    \subsection{Pulsos de sincronismo}
        Os pulsos verticais são formados utilizando o \textsl{Timer 1} no modo \textsl{fast pwm} com \textsl{top} sendo o \textsl{icr1} (\textsl{wgm1[3:0] = 14\(_{10}\)}) e utilizando um \textsl{prescaling} com valor 8 (\textsl{cs1[2:0] = 2\(_{10}\)}), a escolha desse \textsl{Timer} foi feita pelo fato de possuir um contador de 16 \textsl{bits}. O \textsl{timer} possui 3 registradores de comparação (\textsl{ocr1a}, \textsl{ocr1b} e \textsl{icr1}), sendo dois de saída, para poder realizar a construção do pulso, utilizando o modo de comparação para quando o contador atingir o valor de \textsl{ocr1a} a saída ir para nível lógico baixo e quando voltar para \textsl{bottom} ir para nível lógico alto (\textsl{com1a[1:0] = 2\(_{10}\)}) e o terceiro, de entrada, para determinar o inicio da área visível da tela.

        \begin{figure}[H]
    \centering
    % https://wiki.physik.uzh.ch/cms/latex:tikz:electromagnetic_wave
    % https://tex.stackexchange.com/questions/550352/how-to-draw-a-simple-sine-square-sawtooth-waveform-in-latex
    \begin{tikzpicture}[line cap = round, line join = round, quotas/.style={very thin, |<->|}]
        %
        \def\Period{0.6\linewidth}
        \def\RampAmplitude{2.0}
        %
        \def\VericalOne{0.13 * \Period}
        \def\VericalTwo{0.88 * \Period}
        \def\VericalThree{\Period}

        % ramp
        \begin{scope}[local bounding box = ramp]
            \node [anchor = east, text width = 2cm, font = \small\itshape\bfseries] at (0, -\RampAmplitude) {tcnt1};
            \foreach \it/\txt in {0/icr1, -0.24/ocr1a, -1.75/ocr1b, -2/bottom}
            {
                \draw [very thin, dotted] (0, \it) -- ++(1.16 * \Period, 0);
                \node [anchor = east, font = \footnotesize\itshape] at (0, \it) {\txt};
            }
   
            \draw (0, -\RampAmplitude) {-- ++(\Period, \RampAmplitude) -- ++(0, -\RampAmplitude) -- ++(0.16 * \Period, 0.16 * \RampAmplitude)} ;
        \end{scope}
        % pulse
        \begin{scope}[yshift = -2.5cm, local bounding box = pulse]
            \node [anchor = east, text width = 2cm, font = \small\itshape\bfseries] at (0, 0) {oc1a};
            \draw (0, 0) {-- ++(\VericalTwo, 0) -- ++(0, -1) -- ++(\Period - \VericalTwo, 0) -- ++(0, 1) -- ++(0.16 * \Period, 0)};
        \end{scope}

        \def\lastx{0}
        \foreach[count = \i, remember = \it as \lastxit] \it/\initpos in {\VericalOne/-1.75, \VericalTwo/-0.24,\VericalThree/0}
        {
            \draw [very thin, dotted] (\it, \initpos) -- (\it,   -4);
            \draw [very thin, dotted] (\it, \initpos) -- (\it,   -4);
            \draw [quotas] (\lastxit, -4) -- node[inner sep = 1pt, fill = white, font = \scriptsize] {\i} (\it,  -4);
        }
    \end{tikzpicture}
    \caption{Funcionamento do \textit{Timer 1} (\textit{Vsync})}
    \label{fig:timer-1}
\end{figure}

        Na figura é possível ver o funcionamento esperado da saída \textsl{oc1a} em função do contador \textsl{tcnt1}, as regiões 1, 2 e 3 são respectivamente, \textsl{Vertical Back Porch}, \textsl{Vertical Visible Area} junto com a \textsl{Vertical Front Porch} e \textsl{Vertical Sync Pulse}. % TODO: deixar em portugues

        Os valores de comparação da contagem são dados a seguir:
        
        \begin{table}[H]
            \centering
            \begin{tabular}{l l}           \hline
                \textsl{icr1}   & 52499 \\ \hline
                \textsl{ocr1a}  & 52299 \\ \hline
                \textsl{ocr1b}  & 3199  \\ \hline
            \end{tabular}
            \caption{Valores dos registradores de comparação do \textsl{Timer 1}}
            \label{tab:val_timer_1_Hsync}
        \end{table}

        O valor do \textsl{ocr1b} foi calculado para uma linha a menos, para evitar conflitos com os pulsos horizontais, pois quando o seu valor é atingido ocorre uma interrupção para um função que ativa a interrupção para o \textsl{Timer 0} em \textsl{ocr0a}, que será desativada dentro do próprio algoritmo da função chamada pela comparação.

        Em que \textsl{icr1}, por exemplo, é calculado da seguinte forma:
        
        \begin{equation}
            \frac{800 \cdot 525}{8} - 1 = 52499
        \end{equation}
        
        Pelo fato do \textsl{clock} utilizado ser de \(25{,}175 MHz\), igual a frequência dos \textsl{pixels} para o padrão de resolução utilizado, os valores de comparação são numericamente iguais menos um a contagem dos \textsl{pixels}, pelo fato da contagem iniciar em zero, considerando o valor do \textsl{prescaling}.
        
        Os pulsos horizontais são formados pelo \textsl{Timer 0}, deixando o \textsl{Timer 2} livre, no modo \textsl{fast pwm} com \textsl{top} sendo o \textsl{ocr0a} (\textsl{wgm0[2:0] = 7\(_{10}\)}) e utilizando \textsl{prescaling} com valor 8 (\textsl{cs0[2:0] = 2\(_{10}\)}). Esse \textsl{Timer} possui dois registradores de comparação (\textsl{ocr0a}, \textsl{ocr0b}) utilizados para criar um pulso de polaridade negativa entre eles.

        Durante o \textsl{overflow} do contador ocorre uma chamada de interrupção que faz o tratamento da região horizontal \textsl{Back Porch} e da área visível, realizando o envio dos \textsl{pixels} e durante a região \textsl{Front Porch} e \textsl{Sync Pulse} ocorre a construção do \textsl{buffer} da próxima linha.
        
        \begin{figure}[H]
    \centering
    % https://wiki.physik.uzh.ch/cms/latex:tikz:electromagnetic_wave
    % https://tex.stackexchange.com/questions/550352/how-to-draw-a-simple-sine-square-sawtooth-waveform-in-latex
    \begin{tikzpicture}[line cap = round, line join = round, quotas/.style={very thin, |<->|}]
        %
        \def\Period{0.6\linewidth}
        \def\RampAmplitude{2.0}
        %
        \def\VericalOne{0.75 * \Period}
        \def\VericalTwo{\Period}

        % ramp
        \begin{scope}[local bounding box = ramp]
            \node [anchor = east, text width = 2cm, font = \small\itshape\bfseries] at (0, -\RampAmplitude) {tcnt0};
            \foreach \it/\txt in {0/ocr0a, -0.5/ocr0b, -2/bottom}
            {
                \draw [very thin, dotted] (0, \it) -- ++(1.16 * \Period, 0);
                \node [anchor = east, font = \footnotesize\itshape] at (0, \it) {\txt};
            }
   
            \draw (0, -\RampAmplitude) -- ++(\Period, \RampAmplitude) -- ++(0, -\RampAmplitude) -- ++(0.16 * \Period, 0.16 * \RampAmplitude);

        \end{scope}
        % pulse
        \begin{scope}[yshift = -2.5cm, local bounding box = pulse]
            \node [anchor = east, text width = 2cm, font = \small\itshape\bfseries] at (0, 0) {oc0b};
            \draw (0, 0) {-- ++(\VericalOne, 0) -- ++(0, -1) -- ++(\Period - \VericalOne, 0) -- ++(0, 1) -- ++(0.16 * \Period, 0)};
        \end{scope}

        \def\lastx{0}
        \foreach[count = \i, remember = \it as \lastxit] \it/\initpos in {\VericalOne/-0.5,\VericalTwo/0}
        {
             \draw [very thin, dotted] (\it, \initpos) -- (\it,   -4);
             \draw [very thin, dotted] (\it, \initpos) -- (\it,   -4);
             \draw [quotas] (\lastxit, -4) -- node[inner sep = 1pt, fill = white, font = \scriptsize] {\i} (\it,  -4);
        }
    \end{tikzpicture}
    \caption{Funcionamento do \textit{Timer 0} (\textit{Hsync})}
    \label{fig:timer-0}
\end{figure}

        Na figura é possível ver o funcionamento esperado da saída \textsl{oc0b} em função do contador \textsl{tcnt0}, as regiões 1 e 2 são respectivamente, horizontal \textsl{Back Porch} junto com a \textsl{Visible Area} e a \textsl{Front Porch}, e a \textsl{Sync Pulse}.

        Os valores de comparação da contagem são dados a seguir:
        
        \begin{table}[H]
            \centering
            \begin{tabular}{l l}        \hline
                \textsl{ocr0a}  & 99 \\ \hline
                \textsl{ocr0b}  & 87 \\ \hline
            \end{tabular}
            \caption{Valores dos registradores de comparação do \textsl{Timer 0}}
            \label{tab:val_timer_0}
        \end{table}

        Em que o calculo é similar ao realizado para o comparadores do pulso vertical, como é possível ver a seguir para o \textsl{ocr0a}, por exemplo:
    
        \begin{equation}
             \frac{800}{8} - 1 = 99
        \end{equation}

    \subsection{Varredura dos \textsl{pixels}}
        Para realizar a varredura dos \textsl{pixels} foi utilizado o protocolo de comunicação \textsl{usart} no modo master \textsl{spi} do microcontrolador, pelo fato de ser possível fazer uma comunicação de no máximo duas vezes mais lenta que o \textsl{clock} do sistema de forma autônoma, e por ter a mesma frequência dos \textsl{pixels} implica que a resolução horizontal é reduzida pela metade, porém ocupando a mesma quantidade de \textsl{pixels}.
        
        A comunicação é feita pela conexão \verb|txd| e o \verb|hardware| utiliza a conexão \textsl{xck} como \textsl{clock} de sincronismo, mesmo que ele não seja utilizado.
        
\section{\textsl{Pong}}

    O jogo \textsl{Pong} se baseia em lançar uma bola de um lado ao outro da tela com uma "raquete" (\textsl{paddle}) tentando fazer com que o outro jogador erre a bola assim ganhando um ponto, parecido com o \textsl{ping pong}, porém esse projeto se baseia em apenas um jogador, portanto o outro \textsl{paddle} apenas segue a bola, e os elementos do jogo possuem algumas iterações com entre si, por exemplo sempre que a bola atinge o \textsl{paddle} da esquerda ela aumenta a velocidade além de possuir trajetórias diferentes dependendo de onde acertar.

    A disposição dos elementos do jogo é apresentada na figura \ref{fig:game_screen}, onde é possível observar os dois \textsl{paddles}, a bola, a linha divisória central e a pontuação de cada jogador.
    
    \begin{figure}[H]
\centering
\resizebox{0.45\textwidth}{!}
{
\begin{tikzpicture}[x = 1mm, y = 1mm]
    
    \def\width  {640}
    \def\height {480}

    % screen
    \path [fill = black] (1, 1) rectangle ++(\width, \height);
    % paddle 1
    \path [fill = white] (1 - 1, \height / 2 - 32) rectangle ++(8 - 1, 63 - 1);
    % paddle 2
    \path [fill = white] (\width - 8 + 1, \height / 2 - 32) rectangle ++(8, 63);
    % ball
    \path [fill = white] (160, 320) rectangle ++(8, 8);
    % Number 1
    \path [fill = white] (\width / 4 - 16, \height - 32) 
        rectangle ++(32 - 1,   8 - 1)
        rectangle ++(8 - 1,  -64 + 1)
        rectangle ++(-32 + 1, -8 + 1)
        rectangle ++(-8 + 1,  64 - 1);
    % Number 2
    \path [fill = white] (3 * \width / 4 - 16, \height - 32) 
        rectangle ++(32 - 1,   8 - 1)
        rectangle ++(8 - 1,  -64 + 1)
        rectangle ++(-32 + 1, -8 + 1)
        rectangle ++(-8 + 1,  64 - 1);
        
    % middle line
    \foreach \i in {1, 16, ..., \height}
         \path [fill = white] (\width / 2 - 4, \i + 3) rectangle ++(4 - 1, 8 - 1);

\end{tikzpicture}
}
\caption{Janela do jogo}
\label{fig:game_screen}
\end{figure}

    As dimensões de cada elemento da janela segue os valores apresentados na tabela \ref{tab:dimension_screen_game}, para o tracejado central esse é o valor de cada traço:
    
    \begin{table}[H]
        \centering
        \begin{tabular}{l c c}                                   \\ \cline{2-3}
                              & H (\textsl{Pixels}) & V (Linhas) \\ \hline
            Janela            & 640                 & 480        \\ 
            \textsl{Paddle}   & 8                   & 63         \\ 
            Bola              & 8                   & 8          \\
            Tracejado central & 4                   & 6          \\ \hline
        \end{tabular}
        \caption{Dimensão de cada objeto, em que H é o tamanho horizontal e V é o tamanho vertical.}
        \label{tab:dimension_screen_game}
    \end{table}

    Como já apresentado, a bola possui trajetórias diferentes dependendo de onde ela atinge o \textsl{paddle}, quanto mais afastado do centro maior será o ângulo entre a normal e a trajetória refletida, na mesma direção de crescimento do \textsl{paddle}, em que no centro esse valor é igual a zero, como é possível ver na figura \ref{fig:collisions_paddle}:
    
    \begin{figure}[H]
\centering
    % 0b11_111
\begin{tikzpicture}[x = 1mm, y = 1mm, scale = .6]
    \begin{scope}[xshift = 20mm, local bounding box = paddle]
        \draw [black] (0, 0) rectangle ++(8, 63);

        \draw [black] (0, 7 * 63 / 8) -- ++(8, 0);
        \draw [black] (0, 6 * 63 / 8) -- ++(8, 0);
        \draw [black] (0, 5 * 63 / 8) -- ++(8, 0);
        \draw [black] (0, 4 * 63 / 8) -- ++(8, 0);
        \draw [black] (0, 3 * 63 / 8) -- ++(8, 0);
        \draw [black] (0, 2 * 63 / 8) -- ++(8, 0);
        \draw [black] (0, 1 * 63 / 8) -- ++(8, 0);
    \end{scope}

    \begin{scope}[xshift = 0, local bounding box = lines]
        % 3
        \draw [gray, -, dashed]  (0,  8 * 63 / 8 - 4) -- ++(20,  0);
        \draw [black, ->]  (20, 8 * 63 / 8 - 4) -- ++(-20, 15);
        % 2
        \draw [gray, -, dashed]  (0,  7 * 63 / 8 - 4) -- ++(20,  0);
        \draw [black, ->]  (20, 7 * 63 / 8 - 4) -- ++(-20, 7);
        % 1
        \draw [gray, -, dashed]  (0,  6 * 63 / 8 - 4) -- ++(20,  0);
        \draw [black, ->]  (20, 6 * 63 / 8 - 4) -- ++(-20, 3);
        % 0 
        \draw [gray, -, dashed]  (0,  5 * 63 / 8 - 4) -- ++(20,  0);
        \draw [black, <-] (0, 5 * 63 / 8 - 4)  -- ++(20,  0);
    \end{scope}

    \draw [gray, fill opacity = 0.5] (20, 4 * 63 / 8 + 2) -- ++(0, -4);
    \draw [gray, fill opacity = 0.5] (18, 4 * 63 / 8)     -- ++(4, 0);
    
\end{tikzpicture}
\caption{Colisões no \textit{paddle} (as setas funcionam de forma simétrica)}
\label{fig:collisions_paddle}
\end{figure}

    \bibliographystyle{unsrt}
    \bibliography{ref}

\end{document}
